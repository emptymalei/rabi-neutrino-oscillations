% ****** Start of file apssamp.tex ******
%
%   This file is part of the APS files in the REVTeX 4.1 distribution.
%   Version 4.1r of REVTeX, August 2010
%
%   Copyright (c) 2009, 2010 The American Physical Society.
%
%   See the REVTeX 4 README file for restrictions and more information.
%
% TeX'ing this file requires that you have AMS-LaTeX 2.0 installed
% as well as the rest of the prerequisites for REVTeX 4.1
%
% See the REVTeX 4 README file
% It also requires running BibTeX. The commands are as follows:
%
%  1)  latex apssamp.tex
%  2)  bibtex apssamp
%  3)  latex apssamp.tex
%  4)  latex apssamp.tex
%
\documentclass[%
reprint,
%superscriptaddress,
%groupedaddress,
%unsortedaddress,
%runinaddress,
%frontmatterverbose, 
%preprint,
%showpacs,preprintnumbers,
%nofootinbib,
%nobibnotes,
%bibnotes,
 amsmath,amssymb,
 aps,
%linenumbers,
%pra,
%prb,
%rmp,
%prstab,
%prstper,
%floatfix,
]{revtex4-1}

\usepackage{graphicx}% Include figure files

\usepackage{subcaption}

\usepackage{dcolumn}% Align table columns on decimal point
\usepackage{bm}% bold math
%\usepackage{hyperref}% add hypertext capabilities
%\usepackage[mathlines]{lineno}% Enable numbering of text and display math
%\linenumbers\relax % Commence numbering lines

%\usepackage[showframe,%Uncomment any one of the following lines to test 
%%scale=0.7, marginratio={1:1, 2:3}, ignoreall,% default settings
%%text={7in,10in},centering,
%%margin=1.5in,
%%total={6.5in,8.75in}, top=1.2in, left=0.9in, includefoot,
%%height=10in,a5paper,hmargin={3cm,0.8in},
%]{geometry}


%%%%%%%%%%%%%%%%%%%%%%%%%%%
%%%%%%  PREAMBLES %%%%%%%%%
\newcommand{\ud}[1]{{#1^{\dagger}}}
\newcommand{\bra}[1]{\left\langle #1\right|}
\newcommand{\ket}[1]{\left| #1\right\rangle}
\newcommand\Tr{\mathrm{Tr}}
\newcommand{\braket}[2]{\langle #1 \mid #2 \rangle}
\newcommand\I{\mathbb{I}}
\newcommand{\avg}[1]{\left< #1 \right>}
\newcommand{\sech}[1]{{\operatorname{sech}{#1}}}
\newcommand{\csch}[1]{{\operatorname{csch}{#1}}}
%%%%%%  PREAMBLES %%%%%%%%%
%%%%%%%%%%%%%%%%%%%%%%%%%%%

\begin{document}

%\preprint{APS/123-QED}

\title{Parametric neutrino flavor conversions and Rabi oscillations}% Force line breaks with \\
%\thanks{A footnote to the article title}%

\author{Lei Ma}
\email{leima@unm.edu}
\author{Shashank Shalgar}%
\email{shashankshalgar@unm.edu}
\author{Huaiyu Duan}%
\email{duan@unm.edu}
\affiliation{%
 Department of Physics \& Astronomy, University of New Mexico,
 Albuquerque, NM 87131, USA
}%
% \author{Huaiyu Duan}%
%  \email{Second.Author@institution.edu}
% \affiliation{%
%  Authors' institution and/or address\\
%  This line break forced with \textbackslash\textbackslash
% }%

% \collaboration{MUSO Collaboration}%\noaffiliation

% \author{Shashank Shalgar}
%  \homepage{http://www.Second.institution.edu/~Charlie.Author}
% \affiliation{
%  Second institution and/or address\\
%  This line break forced% with \\
% }%
% \affiliation{
%  Third institution, the second for Charlie Author
% }%
% \author{Delta Author}
% \affiliation{%
%  Authors' institution and/or address\\
%  This line break forced with \textbackslash\textbackslash
% }%

% \collaboration{CLEO Collaboration}%\noaffiliation

% \author{Huaiyu Duan}
%  \homepage{http://www.Second.institution.edu/~Charlie.Author}
% \affiliation{
%  Second institution and/or address\\
%  This line break forced% with \\
% }%
% \affiliation{
%  Third institution, the second for Charlie Author
% }%
% \author{Delta Author}
% \affiliation{%
%  Authors' institution and/or address\\
%  This line break forced with \textbackslash\textbackslash
% }%

% \collaboration{CLEO Collaboration}%\noaffiliation



\date{\today}% It is always \today, today,
             %  but any date may be explicitly specified

\begin{abstract}

We develop an interpretation of neutrino flavor conversions in fluctuating matter density based on Rabi oscillations. The interference as a result of multi-frequency matter density fluctuations is also explored and criteria of interference is proposed and verified numerically. Neutrino flavor conversion in matter with multiple-frequency fluctuations is expanded into superposition of Rabi oscillations to explain the relation between neutrino flavor conversion and Rabi oscillations precisely as well as providing more quantitative criteria on the interference between different frequencies of Rabi oscillations.

% \fbox{ABSTRACT PLACEHOLDER}
% \begin{description}
% \item[Usage]
% Secondary publications and information retrieval purposes.
% \item[PACS numbers]
% May be entered using the \verb+\pacs{#1}+ command.
% \item[Structure]
% You may use the \texttt{description} environment to structure your abstract;
% use the optional argument of the \verb+\item+ command to give the category of each item. 
% \end{description}
\end{abstract}

% \pacs{Valid PACS appear here}% PACS, the Physics and Astronomy
                             % Classification Scheme.
%\keywords{Suggested keywords}%Use showkeys class option if keyword
                              %display desired
\maketitle

%\tableofcontents

\section{\label{introduction}Introduction}

In many astrophysical environments, such as core-collapse supernova,  neutrinos propagate through dense fluctuating medium, which will interact with neutrinos and dramatically change the flavor oscillations of neutrinos. Meanwhile, neutrinos deposit energy in matter during the interaction, where the amount of energy deposited depends on the flavor of neutrinos. The significance of matter effect on neutrino flavor oscillations has been demonstrated in Mikheyev–-Smirnov–-Wolfenstein (MSW) effect~\cite{Mikheev:1986gs,wolf78,wolfensteinprd1979}, which is used to explain the deficit of electron flavor neutrino flux, thus solved the solar neutrino problem~\cite{Petcov2002,kuo1989}. Later developments on the theories about matter effect revealed the parametric resonance of neutrino flavor oscillations due to fluctuations in matter density~\cite{Krastev1989,Akhmedov2000}, which is the neutrino analog of transitions between energy levels as a result of external optical stimulation. Parametric resonance is different from the MSW effect since it involves the parameters of the matter density, which is usually the period of matter density fluctuations. Neutrinos passing through inside the Earth can experience parametric resonance~\cite{Akhmedov1999, Petcov1998b}. 

As one of the most intense neutrino sources, supernova neutrinos experience turbulent matter density as they propagate through out the explosion~\cite{Muller2015, Couch2015}, where the flavor conversion is modified by interaction with matter. Meanwhile with neutrinos depositing energy into the shock, neutrino flavor conversion is crucial to the shock evolution of supernova explosion. The turbulent matter density environment for neutrino flavor conversion has been researched~\cite{Loreti1994, Friedland2006,Kneller2010}. Recently matter stimulated neutrino flavor conversion in varying matter density has been researched using Jacobi-Anger expansion by Kneller, et al.~\cite{Kneller2013,Patton2014}. They have shown that many stimulated neutrino flavor conversions can be explained by resonances of the system.

In this paper, we take a step further and interpret parametric resonance~\cite{Akhmedov2000, Krastev1989} as well as other stimulated neutrino flavor conversions~\cite{Kneller2013, Patton2014}, as superposition of Rabi oscillations. In Sec.~\ref{sec:background}, we define the formalism of neutrino flavor conversions in matter used in this paper where the equation of motion and Hamiltonian for neutrino flavor conversion in matter are explicitly written. In Sec.~\ref{sec:simple} we discuss how neutrino flavor conversions are related to Rabi oscillations by neglecting less important terms and show that the system can be reduced to Rabi oscillations systems. Single frequency matter profile, castle wall matter profile are calculated and interpreted as Rabi oscillations. Interference effect between different frequencies are explained using two-frequency matter profile and three-frequency matter profile. In Sec.~\ref{sec:jacobi} we discuss the technique of decomposing the neutrino flavor conversions into summation of Rabi oscillations, by applying a specific unitary transformation and the Jacobi-Anger expansion. In this section, we explain the reason why the approximations in Sec.~\ref{sec:simple} can be done.


% \fbox{TODO:more about oscillation in matter}

\section{\label{sec:background}Background and Formalism}

% \fbox{TODO: Explain why two flavor} didn't add this because the three flavor is really very different.

We consider two-flavor oscillation scenario, in which neutrinos have energy $E$ and mass-squared difference $\delta m^2$ between two mass states propagate through matter which is define by electron number density profile $n(r)$ along the path of neutrino propagation $r$. Three different bases are used in this work, i.e., flavor basis, in which the wave function describes the probability amplitude of different flavors, background matter basis, in which the wave function describe the probability amplitude of different mass states with the appearance of only background matter density, and Rabi basis which will be defined explicitly later. The dynamics of neutrino flavor conversion is determined by Schr\"{o}dinger equation, in which the Hamiltonian in flavor basis is
\begin{equation}
    H^{(\mathrm{f})} =  \frac{1}{2} \begin{pmatrix}
    -\omega_{\mathrm{v}} \cos 2\theta_{\mathrm{v}} + \lambda(r) & \omega_{\mathrm{v}}\sin 2\theta_{\mathrm{v}} \\
   \omega_{\mathrm{v}} \sin 2\theta_{\mathrm{v}} & \omega_{\mathrm{v}} \cos 2\theta_{\mathrm{v}} - \lambda(r)
    \end{pmatrix},
\end{equation}
where $\lambda(r)= \sqrt{2}G_{\mathrm F} n(r)$ is the potential of neutrino interaction with matter and $G_{\mathrm F}$ is the Fermi constant, $\theta_{\mathrm{v}}$ is the vacuum mixing angle, $\omega_{\mathrm{v}} = \delta m^2/2E$ is the vacuum oscillation frequency. For the convenience of notation, we use Pauli matrices $\sigma_i$ to denote the Hamiltonian, so that the Schr\"{o}dinger equation becomes,
\begin{equation}
    \mathrm i\frac{\mathrm d}{\mathrm d r}\Psi(r) = \frac{1}{2} \left(  
    (- \omega_{\mathrm{v}}\cos 2\theta_{\mathrm{v}} + \lambda(r) ) \sigma_3 + \omega_{\mathrm{v}}\sin 2\theta_{\mathrm{v}} \sigma_1 
    \right)
    \Psi(r),
\end{equation}
where $\Psi(r)$ is the wave function in flavor basis. For two flavor scenario, it is written as $ \Psi(r) = \left(
    \psi_{e} ,
    \psi_{x}
    \right)^{\mathrm{T}}$ 
where $\psi_{e}$ and $\psi_{x}$ are the amplitudes for electron flavor and the second flavor ($\mu$ flavor or $\tau$ flavor) respectively. The equation of motion in flavor basis governs the transitions between different flavors.


The potential $\lambda(r)$ for a general perturbation on top of a constant matter profile can be written as
\begin{equation}
    \lambda(r) = \lambda_0 + \delta \lambda(r).
    \label{eq-general-matter-profile}
\end{equation}
For better understanding of the transition between states as a consequence of the fluctuation of matter density, we use the background matter basis, in which the Hamiltonian is diagonalized in the absence of perturbation $\delta\lambda(r)$, so that the Hamiltonian reads
\begin{widetext}
\begin{equation}
    H^{(\mathrm{m})} = -\frac{\omega_m}{2} \sigma_3 + \frac{1}{2} \delta\lambda(r) \cos 2\theta_{\mathrm m} \sigma_3 
     - \frac{1}{2} \delta\lambda(r) \sin 2\theta_{\mathrm m} \sigma_1,
    \label{eq-hamiltonian-bg-matter-basis-general}
\end{equation}
\end{widetext}
where $\theta_{\mathrm m}$ is the mixing angle in a constant matter profile $\lambda_0$, which is calculated using relation
\begin{equation*}
\tan 2\theta_{\mathrm{m}}=\sin 2\theta_{\mathrm v}/\left( \cos 2\theta_{\mathrm v} - \lambda_0/\omega_{\mathrm v} \right)
\end{equation*}
with $\omega_{\mathrm v}$ denoting the vacuum oscillation frequency and $\theta_{\mathrm v}$ denoting the vacuum mixing angle. The energy split is defined as
\begin{equation}
\omega_{\mathrm{m}} = \omega_{\mathrm{v}} \sqrt{ ( \lambda_0/\omega_{\mathrm{v}} - \cos (2\theta_{\mathrm{v}}) )^2 + \sin^2(2\theta_{\mathrm{v}}) }.
\end{equation}




%%%%%%%%%%%%%%%%%%%%%%%%%%%%%%%%%%%%%%%%%%%%%%%%%%%%
%% Simplified Model
%%%%%%%%%%%%%%%%%%%%%%%%%%%%%%%%%%%%%%%%%%%%%%%%%%%%

\section{\label{sec:simple}Parametric Resonance and Rabi oscillation --- a simplified model}%


In this section we present a simple picture to explain neutrino parametric resonance in matter by utilizing the theory of Rabi oscillations which have been well studied in quantum optics~\cite{Boyd2008}. In Appendix~\ref{sec:rabi-oscillations}, we derive the Rabi oscillation transition probabilities using neutrino flavor isospin method introduced in Ref.~\onlinecite{Duan2006a}.




%%%%%%%%%%%%%%%%%%%%%%%%%%%%%%%%%%%%
%%%%%%%%% Rabi oscillation
%%%%%%%%%%%%%%%%%%%%%%%%%%%%%%%%%%%%


% The Hamiltonian Eq.~\ref{eq-hamiltonian-bg-matter-basis-general} has a lot in common with the Rabi oscillation Hamiltonian. One example of Rabi oscillation is the transition between low energy state and high energy state in two level atoms, when stimulated by optical fields with frequency that matches the energy gap of the two level atoms. As the frequency of optical field approaches the energy split of atom, large transition occurs, which is described by Rabi formula. 

% In the light of Rabi oscillations, we argue that the second term in the neutrino system described by Eq.~(\ref{eq-hamiltonian-bg-matter-basis-general}), i.e., the varying diagonal elements of Hamiltonian, is not relevant when the system is close to resonance, as we will explain later using Jacobi-Anger expansion. 


\subsection{\label{sec:single-freq}Single Frequency Matter Profile}


The first example we examine is single frequency matter profile $\delta\lambda(r) = \lambda_1 \sin(k_1 r)$. As will be proved later, the varying $\sigma_3$ term $\delta\lambda(r) \cos 2\theta_{\mathrm m} \sigma_3/2$ in Hamiltonian Eq.~(\ref{eq-hamiltonian-bg-matter-basis-general}), which is the varying energy gap due to varying matter density fluctuations, has little effect on the transition probabilities when the system is close to resonance. With the varying $\sigma_3$ term removed, it indicates that this single frequency matter perturbation neutrino flavor conversion system is reduced to a Rabi oscillation system, with external driving field frequency $k_1$ and energy gap $\omega_{\mathrm m}$. By neglecting the off-resonance terms, the Hamiltonian can be simplified,
% \begin{align}
% H^{(\mathrm{m})}_1 &= -\frac{\omega_{\mathrm m}}{2} \sigma_3 + \frac{1}{2} \lambda_1 \sin(k_1 r) \cos 2\theta_{\mathrm m} \sigma_3 -\frac{1}{2}\lambda_1 \sin(k_1 r) \sin 2\theta_{\mathrm m} \sigma_1 \\
% &\to -\frac{\omega_{\mathrm m}}{2} \sigma_3  -\frac{1}{2}\lambda_1 \sin(k_1 r) \sin 2\theta_{\mathrm m} \sigma_1,
%     \label{eq-hamiltonian-bg-matter-basis-single-frequency}
% \end{align}
\begin{align}
H^{(\mathrm{m})} \to & -\frac{\omega_{\mathrm m}}{2} \sigma_3  - \frac{1}{2} \lambda_1 \sin 2\theta_{\mathrm m} \sin( k_1 r ) \sigma_1\label{eq-hamiltonian-bg-matter-basis-single-frequency} \\
\to & -\frac{\omega_{\mathrm m}}{2} \sigma_3  - \frac{1}{2} A_1 \cos ( k_1 r)  \sigma_1 + \frac{1}{2} A_1\sin(k_1 r) \sigma_2,\nonumber 
\end{align}
where
\begin{equation}
A_1 = \frac{\lambda_1 \sin 2\theta_{\mathrm m} }{2i}.
\label{eq-define-a1}
\end{equation}

The resonance condition is determined by matching the energy gap $\omega_{\mathrm m}$ with external driving field frequency $k_1$, i.e., $\omega_{\mathrm m} \sim k_1$. As the system is approaching resonance condition, the transition probability is predicted well using Rabi formula. As defined in Appendix~\ref{sec:rabi-oscillations}, relative detuning $Q=\lvert k_1 - \omega_{\mathrm m} \rvert /\lvert A_1 \rvert$ measures how off-resonance a system is. $Q\to 0$ indicates the system is very close to resonance, while $Q\to \infty$ indicates the system is far away from resonance.

% For a graphical representation of such resonances, decompositions of Hamiltonian onto Pauli matrices are denoted as vectors on , c.f.~Appendix~\ref{sec:rabi-oscillations}. The Hamiltonian is decomposed into three different components, i.e., the constant energy gap on z axis $\vec H_z$, the varying perturbation on z axis $\vec H_{z}'$, and the rotating perturbations in xy plane $\vec H_{\pm}$. The frequency of the rotating component $H_{+}$ in the xy can easily match the energy gap $\omega_{\mathrm{m}}$ even with a varying $\vec H_z'$ component in the Hamiltonian vector, as long as the system is close to resonance.

To show that this conjecture of simplifying neutrino flavor conversions to Rabi oscillations is correct, we calculate transition probabilities of the neutrinos described by Eq.~(\ref{eq-hamiltonian-bg-matter-basis-general}), and Rabi formula Eq.~(\ref{rabi-system-transition-probability}) from the Rabi oscillations described by Eq.~(\ref{rabi-oscillation-single-perturbation}).
In FIG.~\ref{fig-rabiOscillationsNeutrinoCoincidence}, black dots show the transition probability of neutrino flavor conversions in matter, while solid line stands for the result predicted using Rabi formula. The corresponding relative detuning is $Q = 0$ as we have chosen $k_1=\omega_{\mathrm{m}}$, which sets the system on exact resonance and explains the agreement between Rabi formula and the numerical calculation.

For a single-frequency perturbation in matter profile $\lambda(r) =\lambda_1 +  \lambda_1\sin(k_1 r)$, P. Krastev and A. Smirnov concluded that the parametric resonance condition is $\omega_{\mathrm{m}} \sim n k_1$, if instantaneous $\omega_{\mathrm{m,inst}}(r)$ associated with the matter profile at distance $r$ varies slowly~\cite{Krastev1989}. This condition is exactly the Rabi resonance condition when $n=1$, as such condition matches the driving field frequency to the energy split. All the $n>1$ cases in their condition corresponds to the higher modes derived in Sec.~\ref{sec:jacobi}.





\begin{figure}
                \centering
                \includegraphics[width=\columnwidth]{assets/rabiOscillationsNeutrinoCoincidence-single-frequency}
                %rabiOscillationsNeutrinoCoincidence}
                \caption{Single frequency matter profile and Rabi oscillation.  Black dots are the transition probabilities between two background mass eigenstates for the neutrinos with matter perturbation $A_1\sin(k_1 r)$. During the calculation, $\lambda_0$ is set to $0.5$ of the MSW resonance potential $\lambda_{\mathrm{MSW}}=\omega_{\mathrm{v}}\cos 2\theta_{\mathrm v}$ and mixing angle is chosen so that $\sin^2(2\theta_{\mathrm v}) = 0.093$. The solid line is the probabilities predicted using Rabi formula.}
                \label{fig-rabiOscillationsNeutrinoCoincidence}
\end{figure}






\subsection{Castle Wall Matter Profile}



The approach applied to single frequency matter profile also helps with the understanding of multiple frequency matter profile. Since any matter profile can be Fourier decomposed into superposition of many single frequency matter profile, we show an example of another well studied matter profile which is the periodic castle wall matter profile. The potential shown in Fig.~\ref{fig-castlewall-profile-illustration} is defined as,
\begin{equation}
    \lambda(r) = \begin{cases} 
\Lambda_1, &\quad -\frac{X_1}{2}+nX\le r\le \frac{X_1}{2}+nX \\
\Lambda_2, &\quad \frac{X_1}{2}+nX\le r\le \frac{X_1}{2}+\frac{X_2}{2} +nX
\end{cases}
\label{eq-castle-wall-potential}
\end{equation}
where $X_1$ and $X_2$ are the two periods of the matter profile or potential, $X=X_1+X_2$, and $n$ is integer. The parametric resonance condition derived by E. Akhmedov~\cite{Akhmedov2000} is,
\begin{equation}
    \frac{\tan (\omega_{\mathrm m1}X_1/2)}{\tan (\omega_{\mathrm m2}X_2/2)} = - \frac{\cos 2\theta_{\mathrm m2}}{\cos 2\theta_{\mathrm m1}},
\end{equation}
where $\omega_{\mathrm{m}i}$ and $\theta_{\mathrm{m}i}$ are the energy difference and mixing angle for potential $\Lambda_1$ and $\Lambda_2$ respectively.



Even though this castle wall problem is exactly solved, the condition itself is not transparent. In this subsection, we show that such a system is closed related to Rabi oscillations by making appropriate approximation, which will be be specifically proved in Sec.~\ref{sec:jacobi}.

For illustration purpose, we set the profile to be equal period for the two densities so that $X_1=X_2\equiv X/2$. To show that the neutrino flavor conversions in this castle wall matter profiles is related to Rabi oscillation, we decompose the profile using Fourier series,
\begin{equation}
\lambda(r) = \lambda_0 + \sum_{n=1}^{\infty} \lambda_n \cos\left( k_n  r \right),
\label{eq-castle-wall-fourier-expanded}
\end{equation}
where 
\begin{align*}
\lambda_0 &= (\Lambda_1 + \Lambda_2)/2, \\
\lambda_n & = \frac{2}{(2n-1)\pi}  (-1)^n  \left( \Lambda_1 -  \Lambda_2 \right),\\
k_n &= (2n-1)k_0, \\
k_0 &= 2\pi/X.
\end{align*}

To calculate the transitions between two mass states of background matter potential $\lambda_0$, we use the background matter basis with respect to $\lambda_0$, in which the transition is zero when varying matter profile vanishes. The Hamiltonian
\begin{widetext}
\begin{equation}
H^{(\mathrm m)} = - \frac{1}{2}\omega_{\mathrm m} \sigma_3  + \frac{1}{2} \sum_{n=1}^{\infty} \lambda_n \cos 2\theta_{\mathrm m} \cos\left( k_n  r \right)  \sigma_3 - \frac{1}{2} \sum_{n=1}^{\infty} \lambda_n \sin 2\theta_{\mathrm m}  \cos\left( k_n r \right) \sigma_1,
\label{castle-wall-decomposed-hamiltonian}
\end{equation}
\end{widetext}
determines the transitions between the two background matter states.


\begin{figure}
    \centering
    \includegraphics[width=\columnwidth]{assets/castlewall-profile}
    \caption{The castle wall matter potential profile with $X_1=X_2=X/2$.}
    \label{fig-castlewall-profile-illustration}
\end{figure}

% \begin{figure}[!htbp]
%                 \centering
%                 \includegraphics[width=\columnwidth]{assets/castle-wall}
                % \caption{Transition probabilities for castle wall matter profile. 
                % %The black solid line is the transition probabilities for $\lambda_2-\lambda_1=0.09\lambda_0$ calculated using Rabi formula, while black dots are the numerical solution; The red dashed line is the transition probabilities for $\lambda_2-\lambda_1=0.1\lambda_0$ calculated using Rabi formula, while red up-pointing triangles are the numerical solution; The blue dash-dotted line is the transition probabilities for $\lambda_2-\lambda_1=0.2\lambda_0$ calculated using Rabi formula, while blue down-pointing triangles are the numerical solution. 
                % The black dots, and blue triangles are the transition probabilities of castle wall matter profile calculated numerically for $\Lambda_2-\Lambda_1=0.04\Lambda_0$, and $\Lambda_2-\Lambda_1=0.8\Lambda_0$ respectively, while the black solid line, and blue dashed line are the corresponding Rabi formula. During the calculation, the energy of neutrinos is $10\,\mathrm{MeV}$, mass-squared difference is $\delta m^2=2.6\times 10^{-3}\,\mathrm{eV^2}$, and the vacuum mixing angle chosen so that $\sin^2(2\theta_v)=0.093$. The background potential $\Lambda_0$ is chosen so that it's half the MSW resonance potential, $\Lambda_0 = \frac{1}{2}\lambda_{\mathrm{MSW}}=\frac{1}{2}\omega_{\mathrm{v}}\cos 2\theta_{\mathrm v}$, and the base frequency is set to $k_0 = 2\pi/X = \omega_{\mathrm{m}}$.
%                 }
%                 \label{fig-akhmedovOscPlt}
% \end{figure}

%  trim={<left> <lower> <right> <upper>}
%\begin{widetext}
\begin{figure*}
    \centering
    \begin{subfigure}[b]{0.5\textwidth}
        \centering
        \includegraphics[width=\columnwidth]{assets/castle-wall-2}
    \caption{}
    \label{fig-akhmedovOscPlt-a}
    \end{subfigure}%
    ~ 
    \begin{subfigure}[b]{0.5\textwidth}
        \centering
        \includegraphics[width=\columnwidth]{assets/castle-wall-1}
        \caption{}
        \label{fig-akhmedovOscPlt-b}
    \end{subfigure}
    \caption{Transition probabilities for castle wall matter profile. 
    The black dots, and blue triangles are the transition probabilities of castle wall matter profile calculated numerically for $\Lambda_2-\Lambda_1=3.0\times 10^{-3}\pi\Lambda_0$, and $\Lambda_2-\Lambda_1=3.0\pi\Lambda_0$ respectively, while the black solid line, and blue dashed line are the corresponding Rabi formula. During the calculation, the energy of neutrinos is $10\,\mathrm{MeV}$, mass-squared difference is $\delta m^2=2.6\times 10^{-3}\,\mathrm{eV^2}$, and the vacuum mixing angle chosen so that $\sin^2(2\theta_v)=0.093$. The background potential $\Lambda_0$ is chosen so that it's half the MSW resonance potential, $\Lambda_0 = \frac{1}{2}\lambda_{\mathrm{MSW}}=\frac{1}{2}\omega_{\mathrm{v}}\cos 2\theta_{\mathrm v}$, and the base frequency is set to $k_0 = 2\pi/X = \omega_{\mathrm{m}}$.
                 }
    \label{fig-akhmedovOscPlt}
\end{figure*}
%\end{widetext}


In principle, the base frequency $k_0$ which is determined by the total period $X$ can be arbitrary. In this example, we choose a $X$ so that the base frequency $k_0$ matches the energy gap $\omega_{\mathrm{m}}$. Even though multiple perturbation frequencies show up in Eq.~(\ref{castle-wall-decomposed-hamiltonian}), we identify that only the first frequency $n=1$ is the resonance frequency since we are using $k_0=\omega_{\mathrm{m}}$. As an approximation, we drop all other frequencies $n>1$ regarding the fact that they are off-resonance. Thus, similar to single frequency matter profile, the varying $\sigma_3$ terms have limited effects on the transition probabilities in our case, which leads to
\begin{align*}
    H^{(\mathrm m)} \to & - \frac{1}{2}\omega_{\mathrm m} \sigma_3  - \frac{1}{2} \lambda_1 \sin 2\theta_{\mathrm m}  \cos\left( k_0 r \right) \sigma_1\\
    \to & - \frac{1}{2}\omega_{\mathrm m} \sigma_3  - \frac{1}{2} A_1 \cos ( k_0 r) \sigma_1 + \frac{1}{2} A_1 \sin(k_0 r) \sigma_2,
\end{align*}
where
\begin{equation*}
A_1 = \frac{\lambda_1 \sin 2\theta_{\mathrm m} }{2} .
\end{equation*}


FIG.~\ref{fig-akhmedovOscPlt} shows the neutrino transition probabilities between the two matter states for different $\Lambda_2-\Lambda_1$. For small $\Lambda_2-\Lambda_1$, the Rabi formula prediction using the resonance frequency $k_0=\omega_{\mathrm m}$ matches the numerical results. The reason is that the resonance widths of each other frequencies are small compared to the detunings defined by $k_n-\omega_{\mathrm m}$, so that the interference between the resonance mode and all other modes are small. We notice larger deficit between Rabi formula predictions and numerical results for large $\Lambda_2-\Lambda_1$, which is due to the destruction effect of the interference with between modes. More precisely, the resonance in Fig.~\ref{fig-akhmedovOscPlt-a} and resonance destruction effect in Fig.~\ref{fig-akhmedovOscPlt-b} is qualitatively described the comparison of relative detunings of different frequencies. The salient feature of this Fourier series expanded matter profile Eq.~(\ref{eq-castle-wall-fourier-expanded}) is that the width of each frequency decreases as the order $n$ increases while the detuning of each frequency increases. We calculate the relative detuning for each frequency
\begin{equation}
Q_n = \frac{\lvert k_n -\omega_{\mathrm m} \rvert}{ \lvert \lambda_n  \sin 2\theta_{\mathrm m}/2 \rvert } = \frac{2(n-1)(2n-1)\pi \omega_{\mathrm m}}{(\Lambda_2 - \Lambda_1)\sin 2\theta_{\mathrm m}}
\end{equation}
which is quadratic in $n$ and inversely proportional to $\Lambda_2-\Lambda_1$. Table~\ref{tab-q-values-each-mode} lists the relative detuning for each frequency in Fig.~\ref{fig-akhmedovOscPlt-a} and Fig.~\ref{fig-akhmedovOscPlt-b}. Since the higher order relative detunings $Q_n$ for $n\geq 2$ are much larger than 1 for $\Lambda_2-\Lambda_1 = 0.003\pi\Lambda_0$, we expect the term with frequency $k_1$ dominates. In the other situations, the $Q_2$ term is small which indicates a possible interference between $k_2$ and $k_1$ frequency. 
%We perform the approximation Eq.~(\ref{eq-new-q-value-by-adding-new-long-wavelength-mode}) for the Fourier series expanded castle profile and include only the first two frequencies since higher frequencies terms have very large $Q$'s. We find the relative detuning $Q_1$ change from $0$ to $0.001$ and $1$ for $\Lambda_2-\Lambda_1 = 0.003\pi\Lambda_0$ and $\Lambda_2-\Lambda_1 =3.065\pi\Lambda_0$ respectively. A very small relative detuning $Q_1'=0.001$ have almost no effect on the transition amplitude which is what we observe in Fig.~\ref{fig-akhmedovOscPlt-a}, while $Q_1'=1$ will supress the amplitude by half, which mathches what is shown in Fig.~\ref{fig-akhmedovOscPlt-b}.




% Find the data in MMA file `akhmedov-parametric-resonance.nb`
% {0., 6129.81, 20432.7, 42908.7}
% {0., 5.99981, 19.9994, 41.9987}

\begin{table}
\caption{\label{tab-q-values-each-mode}Relative detuning of each frequency.} 
\begin{ruledtabular} 
\begin{tabular}{lll} 
  & $\Lambda_2-\Lambda_1 = 3.0\times 10^{-3}\pi\Lambda_0$ & $\Lambda_2-\Lambda_1 =3.0\pi\Lambda_0$ \\
\hline \\
 $Q_1$ & $0$ &  $0$ \\ 
 $Q_2$ & $6.1\times 10^3$ &  $6.1$ \\ 
 $Q_3$ & $2.0\times 10^4$ &  $2.0\times 10^1$ \\ 
 $Q_4$ & $4.3\times 10^4$ &  $4.3\times 10^1$ 
\end{tabular} 
\end{ruledtabular} 
\end{table}




%%%%%%%%%
%%%% Interference
%%%%%%%%



\subsection{\label{sec:interference-with-long-wavelength-mode}Interference Between Different Frequencies}

% \fbox{
% \parbox{0.9\columnwidth}{
% \begin{itemize}
%     \item Two limits: strong interference regime and low-interference regime
%     \item For strong interference we include multiple modes
%     \item For weak interference, we can interpret the case that one of the matter profile wavelength is much larger than the other. In this case we have a shift of background matter density of the short wavelength perturbation profile.
%     \item Examples. A slight shift in the background density could remove the resonance, which can be quantified.
%     \begin{equation*}
%         a
%     \end{equation*}
% \end{itemize}
% }
% }


We have illustrated the interference effect between different modes in multiple-frequency Rabi oscillations for castle wall profile. In Fig.~\ref{fig-akhmedovOscPlt}, the transition amplitude drops for larger fluctuations $\Lambda_2-\Lambda_1$. Another destruction effect that can be explained more quantitatively is the interference between a mode at resonance and another mode that has longer wavelength. The effect can be interpreted as shift of energy gap due to the long wavelength perturbation. To model the effect, we construct a Rabi oscillation Hamiltonian with two modes of different frequency,
\begin{widetext}
\begin{equation}
H^{(\mathrm{m})}  = -\frac{\omega_{\mathrm{m}}}{2} \sigma_3 - \frac{1}{2} \sum_{n=1}^2  A_n \cos (k_n r) \sigma_1 + \frac{1}{2} \sum_{n=1}^2  A_n \sin (k_n r) \sigma_2 .
\label{eq-hamiltonian-rabi-two-modes-interference}
\end{equation}
\end{widetext}

To show the destruction effect, the Hamiltonian Eq.~(\ref{eq-hamiltonian-rabi-two-modes-interference}) is written as vectors with sigma matrices as the basis,
\begin{equation}
\mathbf H = \begin{pmatrix}
0\\
0\\
\omega_m
\end{pmatrix} + \begin{pmatrix}
A_1 \cos (k_1 r)\\
-A_1 \sin (k_1 r)\\
0
\end{pmatrix} + \begin{pmatrix}
A_2 \cos (k_2 r)\\
-A_2 \sin (k_2 r)\\
0
\end{pmatrix}.
\end{equation}

The three terms are defined as $\mathbf  H_3$, $\mathbf H_1$, and $\mathbf H_2$, where $\mathbf H_3$ is perpendicular to the other two vectors. Assuming $\mathbf H_2$ is the slow rotating field, i.e., $k_2<k_1$, the shift of energy gap is calculated as
\begin{align}
    \omega_{\mathrm{m}}' =& \sqrt{\omega_{\mathrm{m}}^2 + A_2^2 } \\
    \approx & \omega_{\mathrm{m}} + \frac{A_2^2}{2\omega_{m}},
\end{align}
where we keep only first order of Taylor series.
    
To begin with, we choose the first rotating perturbation to satisfy the resonance condition $k_1=\omega_{\mathrm{m}}$, the condition for the second slow rotating field shifting the system out of resonance is that the detuning becomes larger than the resonance width,
\begin{equation}
\lvert \omega_{\mathrm{m}}' - \omega_{\mathrm{m}} \rvert \geq \lvert A_1 \rvert,
\end{equation}
which leads to
\begin{equation}
\lvert A_2 \rvert \geq \sqrt{2\omega_{\mathrm{m}} \lvert A_1 \rvert} \equiv A_{2,\mathrm{Critical}}.
\end{equation}

The relative detuning $Q$ for the resonance frequency also changes from 0 to
\begin{equation}
    Q_1' = \frac{\lvert k_1 - \omega_{\mathrm{m}} \rvert }{ \lvert A_1 \rvert } = \frac{A_2^2}{2\omega_{\mathrm{m}} \lvert A_1 \rvert }.
    \label{eq-new-q-value-by-adding-new-long-wavelength-mode}
\end{equation}
If the conjecture is correct, we expect the transition amplitude to decrease as we have larger $\lvert A_2\rvert$.


\begin{figure}
                \centering
                \includegraphics[width=\columnwidth]{assets/interference-reduction}
                \caption{Reduction of transition amplitudes. Black dots are the numerical transition amplitudes for $A_2=4.5\times 10^{-4}\omega_{\mathrm m}$; Green up-pointing triangles are the numerical transition amplitude for $A_2=A_{2,\mathrm{Critical}}=1.4\times10^{-2}\omega_{\mathrm m}$; Red diamonds are the numerical transition amplitudes for $A_2=2.4\times10^{-2}\omega_{\mathrm m}$. In all the calculations, we choose $A_1=10^{-4}\omega_{\mathrm m}$, $k_1=\omega_{\mathrm m}$, $k_2=0.01\omega_{\mathrm m}$. The lines are the probabilities predicted using Rabi formula correspondingly. During the calculation, $\Lambda_0$ is set to half of the MSW resonance potential, $\Lambda_0 = \frac{1}{2}\lambda_{\mathrm{MSW}}=\frac{1}{2}\omega_{\mathrm{v}}\cos 2\theta_{\mathrm v}$.}
                \label{fig-rabi-oscillations-energy-gap-change}
\end{figure}


We choose a system with two frequencies where the first frequency has amplitude $A_1 = 10^{-4}\omega_{\mathrm{m}}$ and frequency $k_1 = \omega_{\mathrm{m}}$. With a small amplitude of the second frequency, $A_2=4.5\times10^{-4}\omega_{\mathrm{m}}$, we obtain almost full resonance. For larger $A_2$ the destruction effect is more effective, as shown in Fig.~\ref{fig-rabi-oscillations-energy-gap-change}. The agreement between markers and corresponding lines verifies the approximation that the slow rotating field works as a shift in energy gaps. The relative detunings are $0.001$, $1$, and $3$ for $A_2= 2.4\times10^{-4}\omega_{\mathrm{m}}$, $A_2=A_{2,\text{Critical}}=1.4\times10^{-2}\omega_{\mathrm{m}}$, and $A_2= 2.4\times10^{-2} \omega_{\mathrm{m}}$ respectively. We also notice that the width of each cases doesn't change since we kept $A_1$ fixed for each calculation, which indicates that the decreasing in transition amplitude is because of the increasing in detuning.







Fig.~\ref{fig-interference-reduction-three-modes} shows an intriguing phenomenon we observe. The transition probabilities are for the Hamiltonian of three-frequency matter profile
\begin{widetext}
\begin{equation}
H^{(\mathrm m)} =  -\frac{\omega_{\mathrm{m}}}{2} \sigma_3 + \frac{1}{2} \sum_{n=1}^3 \cos 2\theta_{\mathrm m} A_n \sin (k_n r) \sigma_3  - \frac{1}{2} \sum_{n=1}^3 \sin 2\theta_{\mathrm m} A_n \cos (k_n r) \sigma_1 
\label{eq-hamiltonian-rabi-three-frequency-interference}
\end{equation}
\end{widetext}
with different combinations of $A_n$ and $k_n$. The most significant difference from the example for two-frequency interference Eq.~(\ref{eq-hamiltonian-rabi-two-modes-interference}) is that we have varying energy gap shift. The red line shows the step-like increase in transition probability. Each step-like jump is the point where the second frequency matter potential becomes zero so that the resonance is restored. However, by adding another frequency we have eliminated such resonance restored points since the matter potential of the combination of second and third frequency have no zero points within this range of calculation. This is another verification of our conjecture that the destruction of resonance is due to the energy gap shift, or equivalently the relative detuning. The reason we do not observe this phenomenon in Fig.~\ref{fig-akhmedovOscPlt} is because the amplitude of matter potential is decreasing rapidly for for larger $n$'s which leads to a dramatic decrease in relative detuning.


\begin{figure}
    \centering
    \includegraphics[width=\columnwidth]{assets/interference-reduction-three-modes}
    \caption{Resonance destruction due to energy gap shift. The black line is the transition probability for two frequencies $k_1=\omega_{\mathrm m}$, $k_2=\omega_{\mathrm m}/\pi$, $A_1=10^{-4}\omega_{\mathrm m}$, $A_2 = 1.4\times 10^{-2}\omega_{\mathrm m}$; The red line is the transition probability for two frequencies $k_1=\omega_{\mathrm m}$, $k_2=\omega_{\mathrm m}/1000 \pi$, $A_1=10^{-4}\omega_{\mathrm m}$, $A_2 = 1.4\times 10^{-2}\omega_{\mathrm m}$; The green line is the transition probability for three frequencies $k_1=\omega_{\mathrm m}$, $k_2=1/1000\pi$, $k_3=\omega_{\mathrm m}/543\pi$, $A_1=10^{-4}\omega_{\mathrm m}$, $A_2 =A_3 = 1.4\times10^{-2}\omega_{\mathrm m}$. The vertical gridlines are the points where the second matter potential becomes 0.}
    \label{fig-interference-reduction-three-modes}
\end{figure}











% \begin{figure}[!htbp]
%                 \centering
%                 \includegraphics[width=\columnwidth]{assets/interference}
%                 \caption{.... Remake this fig}
%                 \label{fig-interference}
% \end{figure}








%%%%%%%%%%%%%%%%%%%%%%%%%%%%%%%%%%%%%%%%%%%%%%%%%%%%
%%%%%%%%%  Stimulated Neutrino Oscillations  %%%%%%%%%%%%%%%%%%%
%%%%%%%%%%%%%%%%%%%%%%%%%%%%%%%%%%%%%%%%%%%%%%%%%%%%

\section{\label{sec:jacobi}Parametric Resonance and Rabi oscillation --- Jacobi-Anger expansion}


With the intuition of the Rabi oscillation itself as well as the interference with long wave length mode shown in Sec.~\ref{sec:simple}, we can interpret the transition probabilities of any matter profile more precisely if the system can be exactly decomposed into multiple Rabi oscillations. In this section, we show that the matter effect can be decomposed into superpositions of Rabi oscillations by applying a Jacobi-Anger expansion to the Hamiltonian. Similar trick has been used in Ref.~\onlinecite{Kneller2013}. However, we take a different approach by applying a unitary transformation first which make the motivation of Jacobi-Anger expansion, then write the final result as superpositions of Rabi oscillation. For a system with general matter perturbation, c.f. Eq.~(\ref{eq-hamiltonian-bg-matter-basis-general}), we apply an unitary transformation of the form
\begin{equation}
    \mathbf{U} =  \begin{pmatrix} e^{-i \eta (r)} & 0 \\  0 & e^{i \eta (r)}  \end{pmatrix},
    \label{eq-rabi-transformation}
\end{equation}
which is a transformation used in Ref.~\onlinecite{Kneller2006} to remove the diagonal elements of the Hamiltonian. In this work, this transformation is used to remove the varying $\sigma_3$ terms $\delta\lambda(r) \cos 2\theta_{\mathrm m} \sigma_3/2$ in the Hamiltonian, so that the energy gap is fixed in the new basis $\left(\ket{\nu_{\mathrm{r1}}},\ket{\nu_{\mathrm{r2}}}\right)^{\mathrm{T}}$, which is defined as
\begin{equation}
    \begin{pmatrix} \ket{\nu_{\mathrm{r1}}}\\ \ket{\nu_{\mathrm{r2}}} \end{pmatrix} =  \mathbf{U}^\dagger \begin{pmatrix} \ket{\nu_{\mathrm{L}}} \\ \ket{\nu_{\mathrm{H}}} \end{pmatrix}.
    \label{eq-rabi-basis}
\end{equation}
For convenience, we name this unitary transformation Rabi transformation in Eq.~(\ref{eq-rabi-transformation}) as well as the new basis in Eq.~(\ref{eq-rabi-basis}) the Rabi basis. In Rabi basis, we find the Schr\"{o}dinger equation
\begin{widetext}
\begin{equation*}
    \begin{pmatrix}  \frac{\mathrm d\eta}{\mathrm dr}  & 0 \\ 0 & - \frac{\mathrm d\eta}{\mathrm d r}  \end{pmatrix} \begin{pmatrix} \psi_{\mathrm R1} \\ \psi_{\mathrm R2} \end{pmatrix} + \mathrm i \frac{\mathrm d}{\mathrm dr} \begin{pmatrix} \psi_{\mathrm R1} \\ \psi_{\mathrm R2} \end{pmatrix} 
    =
\left[ -\frac{\omega_{\mathrm m} }{2} \sigma_3  + \frac{\delta \lambda}{2} \cos 2\theta_{\mathrm m}  \sigma_3  - \frac{\delta \lambda}{2} \sin 2\theta_{\mathrm m} \begin{pmatrix} 0 & e^{2\mathrm i\eta} \\ e^{-2 \mathrm i\eta } & 0 \end{pmatrix}   \right] \begin{pmatrix} \psi_{\mathrm R1} \\ \psi_{\mathrm R2} \end{pmatrix},
\end{equation*}
\end{widetext}
in which the varying diagonal elements in Hamiltonian can be eliminated by choosing $\eta(x)$ properly, i.e.,
\begin{equation}
    \eta(x) - \eta(0) =  \frac{\cos 2\theta_{\mathrm{m}}}{2} \int_0^x \delta\lambda (\tau) d\tau.
\end{equation}
In Rabi basis, the Schr\"{o}dinger equation becomes
%\begin{widetext}
\begin{equation*}
    i \frac{d}{dx} \begin{pmatrix} \psi_{\mathrm r1} \\ \psi_{\mathrm r2} \end{pmatrix} = \left[ - \frac{\omega_{\mathrm m}}{2} \sigma_3 - \frac{\delta \lambda}{2} \sin 2\theta_m \begin{pmatrix} 0 & e^{2i\eta} \\ e^{-2 i\eta } & 0 \end{pmatrix}\right] \begin{pmatrix} \psi_{\mathrm r1} \\ \psi_{\mathrm r2} \end{pmatrix},
\end{equation*}
%\end{widetext}
where $\eta(x) = - A_1 \cos 2\theta_{\mathrm m} \cos (k_1 x)/(2 k) $ for single frequency matter profile with potential $\delta\lambda(x) = A_1\sin(k_1x)$.  One can easily show that the transition probability between two eigenstates in Rabi basis is the same as the transition probability between two eigenstates in background matter basis.

To make connection with Rabi oscillation, we apply Jacobi-Anger expansion, which is used in Ref.~\onlinecite{Kneller2013}, to decompose the $\exp\left( i z \cos\left(k_1 r \right) \right)$-like term in Hamiltonian into linear combinations of terms that is proportional to $\exp\left(i n k_1 r \right)$, i.e., to decompose spherical waves into plane waves. The decomposed form of Hamiltonian explicitly shows that the Hamiltonian is a summation of Rabi systems, which is
%\begin{widetext}
\begin{equation*}
    H^{(\mathrm{R})} = 
    -\frac{\omega_{\mathrm{m}}}{2} 
    -  \frac{1}{2} \sum_{n=-\infty}^\infty B_n \begin{pmatrix}
    0 &  \Phi_n e^{\mathrm i n k_1  r} \\
     \Phi_n^* e^{ - \mathrm i n k_1 r} & 0
    \end{pmatrix},
\end{equation*}
%\end{widetext}
where
\begin{align*}
    B_n &= \tan 2\theta_{\mathrm m} n k_1 J_{n} \left( \frac{A_1}{k_1}\cos 2\theta_{\mathrm m} \right),\\
    \Phi_n &= e^{i\pi (3n/2+1)},
\end{align*}
with $J_n$ standing for the Bessel function.
The constant phase $\Phi_n$ doesn't play any role for the reason discussed in Appendix~\ref{sec:rabi-oscillations}. Phase in matter potential would also go into $\Phi_n$, for which reason, phase is not included in the current problem. The first term in Hamiltonian describes the energy gap, while the second term is the summation of many driving fields. The resonance width for a given mode $n$ is $\lvert B_{n}\rvert$. It's worth mentioning that for large $n$, we have~\cite{Ploumistakis20092897}
\begin{equation}
J_n(n \sech \alpha) \sim \frac{ e^{n(\tanh\alpha - \alpha)} }{\sqrt{ 2\pi n \tanh \alpha } }.
\end{equation}
It's straightforward to prove that resonance width decreases dramatically for large $n$ thus higher order modes become insignificant.



When the system has one dominate resonance mode and without significant interference between the resonance mode and other modes, all off-resonance modes can be dropped without significantly changing the transition probabilities. However, as we have shown previously in Sec.~\ref{sec:simple}, interference might happen between different modes. A qualitative measure should be introduced so that we know what the conditions are for dropping other modes. The following subsections will determine the important modes of the system (i.e., which $n$ to include) and explore the interference between modes hence explain the coincidence presented in the previous sections. For numerical calculations, we use dimensionless quantities which are scaled using the characteristic energy scale $\omega_{\mathrm{m}}$, e.g.,
\begin{align*}
    \hat r &= \omega_{\mathrm{m}}r, \\
    \hat k_1 & = \frac{k_1}{\omega_{\mathrm{m}}}, \\
    \hat A_1 & = \frac{A_1}{\omega_{\mathrm{m}}}, \\
    \hat B_n &= \frac{B_n}{\omega_{\mathrm{m}}}.
\end{align*}

Through the Rabi transformation and Jacobi-Anger expansion, the varying diagonal elements in Hamiltonian, i.e., varying $\sigma_3$ term, is transformed to a perturbing field with many different frequencies $n k_1$.






\subsection{The Important Factors}


% \fbox{
% \parbox{0.9\columnwidth}{
% \begin{itemize}
%             \item Width of resonance $B$
%             \item Deviation from exact resonance $g$, called {\bf{detuning}} (value).
%             \item Oscillation wavelength of mode (determined by Rabi frequency, which is in turn related to $B$ and $g$ ) compared to size of physical system
% \end{itemize}
% }}


To determine the important modes and justify the approximations used in Sec.~\ref{sec:simple}, the resonance width for each mode which is determined by $B_n$, the deviation from exact resonance (i.e., detuning) which is calculated as $\lvert n k_1 - \omega_{\mathrm{m}} \rvert$, and oscillation wavelength of each mode compared to the size of the physical system should be considered.

% By utilizing the theory of Rabi oscillation, we know that the oscillation wavelength of each mode is determined by the Rabi frequency $\Omega_n$.

More specifically, we define the resonance quality of each mode using the reciprocal of scaled detuning 
 \begin{equation}
 Q_n = \lvert B_n/( n k_1 - \omega_{\mathrm{m}}  )\rvert,
 \end{equation}
which measures how well the resonance is. For exact resonance, $Q_n=0$, while large $Q_n$ says the system is too far away from the resonance of this specific mode. Small $Q_n$ modes are the modes that are dramatically important to the system. Meanwhile, modes that has much larger oscillation wavelength are not subjected to be considered even though their $Q_n$'s are close to zero, which is due to the fact that such modes have not accumulated much transition probabilities within the size of the physical system.


% \begin{figure}[!htbp]
%                 \centering
%                 \includegraphics[width=\columnwidth]{assets/placeholder.jpg}%fig-modes-and-detuning}
%                 \caption{fig-modes-and-detuning}
%                 \label{fig-modes-and-detuning}
% \end{figure}


% \fbox{
% \parbox{0.9\columnwidth}{%
% FIGS:
% \begin{itemize}
%     \item Resonance width and transition amplitude? But this is probably something people already know about.
%     \item {\bf Width is dropping as n increase. Combine the plot with the limit when $n$ is large.}
%     \item Show that only the first mode is important in some situations. And the correction due to higher orders.
%     \item How to show that? The widths are too small to be shown. Maybe stack the together? Or make a plot of width dropping vs n?
% \end{itemize}
% }
% }





% \begin{figure}[!htbp]
%                 \centering
%                 \includegraphics[width=\columnwidth]{assets/placeholder.jpg}%fig-modes-and-detuning}
%                 \caption{fig-important-factors-work: show that the $Q_n$ and the wavelength argument work. By choosing different $Q_n$'s and wave length. Make two figs maybe: one for varying $Q$, the other for varying wavelength. Higher order correction verification is also good.}
%                 \label{fig-important-factors-work}
% \end{figure}




Similar quantities are to be considered for multi-frequency matter profile, such as the resonance width of each mode $B_N$, detuning $\lvert \sum_a n_a k_a - \omega_{\mathrm{m}} \rvert$, and Rabi frequency $\Omega_N$, as well as $Q_N =\lvert B_N/( \sum_a n_a k_a - \omega_{\mathrm{m}}  )\rvert$. 

In principle, we can always approximate the system by including more modes with larger relative detuning while neglecting the modes with wavelength much longer than the size of physical system. Notheless such effort doesn't simplify the calculations.




\subsection{Single Frequency Matter Profile}

For the single frequency matter potential $\lambda = \lambda_0 + \lambda_1 \sin(k_1 r)$ discussed in Sec.~\ref{sec:single-freq}, we removed the varying $\sigma_3$ term by arguing that this term has no effect on transition probabilities when the system is close to resonance, $k_1 \sim \omega_{\mathrm m}$. The reason is that only the first mode $n=1$ is on resonance when $k_1=\omega_{\mathrm m}$ and all other modes are far from resonance, thus
\begin{align}
H^{\mathrm R} \approx & -\frac{\omega_{\mathrm m}}{2}\sigma_3 - \frac{1}{2} B_1 \begin{pmatrix}
0 & \Phi_1 e^{i k_1 r} \\
\Phi_1^* e^{-ik_1r} & 0
\end{pmatrix}\label{eq-single-frequency-first-mode-hamiltonian} \\
\approx & -\frac{\omega_{\mathrm m}}{2} \sigma_3 - \frac{A_1}{2} \cos(k_1 r) \sigma_1 + \frac{A_1}{2} \sin(k_1 r) \sigma_2\nonumber,
\end{align}
where $A_1$ is defined in Eq.~(\ref{eq-define-a1}) and approximation
\begin{equation*}
J_1\left( \frac{\lambda_1}{k_1}\cos (2\theta_{\mathrm m}) \right) \approx \frac{\lambda_1}{2k_1}\cos (2\theta_{\mathrm m})
\end{equation*}
for small $\lambda_1\cos(2\theta_{\mathrm m})/k_1$ is used in the last step. Thus we reach a similar equation to the approximation we used in Sec.~\ref{sec:single-freq}. Small $\lambda_1\cos(2\theta_{\mathrm m})/k_1$ corresponds to small resonance width for Eq.~(\ref{eq-hamiltonian-bg-matter-basis-single-frequency}) and also Eq.~(\ref{eq-single-frequency-first-mode-hamiltonian}) so that the interferences are small. For $\lambda_1 \cos(2\theta_{\mathrm{m}})/k_1\gg 1$, the resonance width is very large compared to $\lvert n k_1 - \omega_{\mathrm{m}}$, which is the distance between different order of resonances, thus bringing in the interference between different modes. The relative detuning for each mode in the Jacobi-Anger expansion for single frequency matter profile used in Fig.~\ref{fig-rabiOscillationsNeutrinoCoincidence} is calculated, which shows that the relative detuning of the first mode is $0$, while the relative detuning of the second and third mode are $1.1\times 10^9$ and $9.7\times 10^{13}$ which are too huge to have any effect on the overall transition probability.







\subsection{Multifrequency Matter Profile}

The interference effect shown in Fig.~\ref{fig-interference-reduction-three-modes} is also interpretable by means of Jacobi-Anger expansion. Table~\ref{tab-q-values-lambda-each-mode-interference} shows the smallest relative detunings and corresponding oscillation wavelength of each mode. As expected, the third frequency introduced many small relative detuning modes, which interferes more compared to the two frequency cases.


% {1,0}	0.
% {0,1}	248.873
% {0,-1}	481.292
% {0,2}	6477.51
% {0,-2}	29173.9
% {0,3}	78458.1
% {-1,0}	103283.
% {1,1}	608730.
% {1,-1}	1.17721*10^6
% {0,-3}	3.40313*10^6

% ,{1,0}	0.
% {1,1}	215.309
% {1,-1}	215.446
% {1,2}	323.895
% {1,-2}	324.307
% {1,4}	590.72
% {1,-4}	592.226
% {1,3}	741.798
% {1,-3}	743.216
% {1,6}	806.351

% ,{1,0,0}	0.
% {1,35,-19}	17.913
% {1,-35,19}	17.9131
% {1,-11,6}	53.225
% {1,11,-6}	53.2267
% {1,37,-20}	63.8707
% {1,-37,20}	63.8775
% {1,-22,12}	111.436
% {1,22,-12}	111.443
% {1,39,-21}	129.027

% }


\begin{table*}
\caption{\label{tab-q-values-lambda-each-mode-interference}Relative detuning and oscillation wavelength of each mode.} 
\begin{ruledtabular} 
\begin{tabular}{lll|lll|lll} 
 \multicolumn{3}{c|}{$k_2=\omega_{\mathrm m}/\pi$} & \multicolumn{3}{c|}{$k_2=\omega_{\mathrm{m}}/1000\pi$} & \multicolumn{3}{c}{three-frequency} \\
\hline
   $\{n_i\}$ & $Q$ & $\lambda$ & $\{n_i\}$ & $Q$ & $\lambda$ & $\{n_i\}$ & $Q$ & $\lambda$  \\
\hline 
 {1,0} &	0.  &     324472.    & {1,0} &	0.  &   $3.32075\times 10^6$       & {1,0,0} &	0. &  $2.29097\times10^7$   \\
{0,1} &	248.873 &   9.217      &  {1,1} &	215.309 &    19739.     &   {1,35,-19} &	17.913  & $2.1229\times 10^6$  \\
{0,-1} &	481.292 &   4.76608    & {1,-1} &	215.446 &    19739.        &  {1,-35,19} &	17.9131 &  $2.1231\times 10^6$  \\
{0,2} &	6477.51 &     17.2909       & {1,2} &	323.895 &   9869.56          & {1,-11,6} &	53.225 &  359495  \\
{0,-2} &	29173.9 &    3.83912     & {1,-2} &	324.307  &    9869.56      &   {1,11,-6} &	53.2267   &      359556  \\
{0,3} &	78458.1  &      139.408       &{1,4}  &	590.72   &     4934.8     &  {1,37,-20}  &	63.8707 & 241920 \\
{-1,0} &	103283. &    3.14159      & {1,-4}  &	592.226  &     4934.8      & {1,-37,20} &	63.8775 & 241992. \\
{1,1} &	608730.  &       19.7392      &  {1,3}  &	741.798  &   6579.73        &  {1,-22,12} &	111.436 &    162934.\\
{1,-1} &	$1.17721\times10^6$  &    19.7392      & {1,-3} &	743.216 &     6579.73        & {1,22,-12} &	111.443  & 163032.\\
\end{tabular} 
\end{ruledtabular} 
\end{table*}









A matter profile as complicated as the castle wall profile can be decomposed into a summation of multiple frequencies. Even for arbitrary multi-frequency matter potential
\begin{equation}
    \lambda(r) = \lambda_0 + \sum_n A_n \sin (n k_n r),
\end{equation}
the Hamiltonian also takes a form of summation of Rabi oscillations,
\begin{widetext}
\begin{equation}
    H^{(\mathrm R)} = -\frac{\omega_{\mathrm m}}{2} - \frac{1}{2} \sum_{n_1=-\infty}^\infty \cdots \sum_{n_N = -\infty}^\infty B_{\{n_i\}} 
    \begin{pmatrix}
    0 & \Phi_{\{n_i\}} e^{i\left( \sum_a n_a k_a \right)r} \\
    \Phi_{\{n_i\}}^* e^{-i\left( \sum_a n_a k_a \right)r} & 0
    \end{pmatrix},
\end{equation}
\end{widetext}
where
\begin{align*}
    B_{\{n_i\}} &=  \tan 2\theta_{\mathrm m} \left( \sum_a n_a k_a \right) \left( \prod_a J_{n_a}\left( \frac{A_a}{k_a}\cos 2\theta_m \right) \right),\\
    \Phi_{\{n_i\}} &= e^{i\pi (3\sum_a n_a/2+1)},
\end{align*}
which is similar to the single frequency matter potential Hamiltonian. A mode is defined by a certain choice of $n_1,\cdots,n_N$. As defined for single frequency matter profile, a relative detuning can also be defined for multi-frequency matter profile
\begin{equation}
    Q_{\{n_i\}} = \frac{\lvert \sum_a n_a k_a - \omega_{\mathrm m} \rvert }{\lvert B_{\{n_i\}} \rvert}.
\end{equation}

As the number of frequencies becomes large, the resonance modes are easily clogged together since many set of $\{n_i\}$'s can be chosen so that the relative detunings are all close to 0. Resonance modes that has oscillation wavelength much larger than the physical system such as supernova explosion are neglected since these modes have not developed large transition probabilities within the size of the physical system. On the other hand, long wavelength modes are the source of resonance destruction for short wavelength resonance modes, as shown in Sec.~\ref{sec:interference-with-long-wavelength-mode}.








\section{\label{conclusions}Conclusions}

%%%% Do not repeat what has been said


In conclusion, we have provided a novel interpretation for neutrino flavor conversion in matter. The first step was to interpret the neutrino flavor conversions in background matter basis. By intuition, we interpreted the neutrino flavor conversion in a single frequency matter profile as Rabi oscillations when the matter fluctuation frequency is close to the energy gap, which is one of the resonance conditions. We anticipated the matter density fluctuations which introduced a varying part in the energy gap have limited effects on neutrino flavor conversions given the matter fluctuation frequency matches the background energy gap, so that the matter fluctuation only works as a pure flipping field that converts neutrinos from one flavor to another. Relative detuning has been the key to this paper since it is critical to quantify how off-resonance a Rabi oscillation is. In the case of single frequency matter profile, the relative detuning becomes $0$ under the resonance condition. We also showed that the interference in Rabi oscillations between a high frequency matter profile, which is on-resonance and a low frequency matter profile, which is off-resonance, is related to the shift in energy gap. Consequently, the strength of interference is measured by modification of relative detuning due to the shift of energy gap. 

Neutrino flavor conversions in matter can also be described in a basis where the oscillations are consequences of superposition of Rabi oscillations, which are modes of oscillations. This view was applied to emphasis the approximations that the change of energy gap due to matter fluctuation can be neglected under resonance condition in the previous background matter basis. The destruction effect due to multiple frequencies matter profile was also explained as the interference between different modes. If only one mode has a close to zero relative detuning while all other modes have much larger relative detunings, only the one with small relative detuning dominates the neutrino flavor conversion. Otherwise, interferences serves as a destruction effect of the neutrino flavor conversion.








\section{\label{acknowledgement}Acknowledgement}

The first author would like to thank J. Kneller and K. Patton for their help during this research. 




%%%%%%%%%%%%%%%%%%%%%%%%%%%%%%%%%%%%%%%%%%
%%%%%%%%%%%%% APPENDIX  %%%%%%%%%%%%%%%%%%
%%%%%%%%%%%%%%%%%%%%%%%%%%%%%%%%%%%%%%%%%%



% \newpage\null\thispagestyle{empty}
% \vspace{20em}
% This page is intentionally left blank
% \newpage


% \clearpage
\appendix
% \section{\label{sec:three-bases}Three Bases}

% \fbox{Define the three different bases used.}

\section{\label{sec:rabi-oscillations}Rabi Oscillations}

%  trim={<left> <lower> <right> <upper>}
%\begin{widetext}
% \begin{figure*}
%     \centering
%     \begin{subfigure}[b]{0.5\textwidth}
%         \centering
%         \includegraphics[trim={2cm 3.2cm 9.5cm 1cm},clip]{assets/rabi-isospin-static-frame}
%     \caption{}
%     \label{fig-rabi-isospin-static-frame}
%     \end{subfigure}%
%     ~ 
%     \begin{subfigure}[b]{0.5\textwidth}
%         \centering
%         \includegraphics[trim={6cm 3cm 9.5cm 2cm},clip]{assets/rabi-isospin-rotating-frame}
%         \caption{}
%         \label{fig-rabi-isospin-rotating-frame}
%     \end{subfigure}
%     \caption{Rabi oscillations in static frame and rotating frame. In both figures the red dashed vector is the flavor isospin, while the black solid vectors are the vectors of Hamiltonian. The left panel shows the rotating Hamiltonian $\mathbf{H}_3+\mathbf{H}_+$. The right panel shows the rotation of flavor isospin around a static vector $\mathbf{H}'_3+\mathbf{H}'_+$ in the rotating frame.}
%     \label{fig-rabi-isospin-different-frame}
% \end{figure*}
%\end{widetext}

\begin{figure}
        \centering
        \includegraphics[width=\columnwidth, trim={20cm 10cm 50cm 10cm},clip]{assets/rabi-isospin-rotating-frame}
    \caption{Rabi oscillations in static frame and rotating frame. In both figures the red dashed vector is the flavor isospin, while the black solid vectors are the vectors of Hamiltonian. The left panel shows the rotating Hamiltonian $\mathbf{H}_3+\mathbf{H}_+$. The right panel shows the rotation of flavor isospin around a static vector $\mathbf{H}'_3+\mathbf{H}'_+$ in the rotating frame.}
    \label{fig-rabi-isospin-rotating-frame}
\end{figure}

In this appendix we introduce flavor isospin \cite{Duan2006a} to Rabi oscillations and derive the transition probabilities as well as explain the resonance and width briefly.


The Hamiltonian for Rabi oscillation is
\begin{equation}
    H_{\mathrm R} = -\frac{\omega_{\mathrm R}}{2}\sigma_3 - \frac{A_{\mathrm{R}} }{2}  \left( \cos(k_{\mathrm{R}} t +\phi_{\mathrm{R}})\sigma_1  - \sin(k_{\mathrm{R}} t +\phi_{\mathrm{R}}) \sigma_2\right),
    \label{rabi-oscillation-single-perturbation}
\end{equation}
in which $\omega_{\mathrm R}$ serves as the energy split of the two level system, while $A_{\mathrm{R}}$ and $k_{\mathrm{R}}$ are the strength and frequency of the driving field, respectively. A decomposition of the second term shows that
\begin{equation*}
H_{\mathrm R} 
= - \frac{\boldsymbol{\sigma}}{2} \cdot (\mathbf{H}_3 + \mathbf{H}_+ ) ,
\end{equation*}
where $\boldsymbol{\sigma}$ is the the vector of Pauli matrices, and the three vectors are
\begin{align}
    \mathbf{H}_3 = & \begin{pmatrix}
    0 \\ 0 \\ \omega_{\mathrm R}
    \end{pmatrix}, \\
    \mathbf{H}_+ = & \begin{pmatrix}
    A_{\mathrm{R}} \cos(k_{\mathrm{R}} t +\phi_{\mathrm{R}}) \\
    - A_{\mathrm{R}} \sin(k_{\mathrm{R}} t +\phi_{\mathrm{R}}) \\
    0
    \end{pmatrix}.
\end{align}

The three vectors are mapped onto a Cartesian coordinate system, so that $\mathbf{H}_3$ is the vector aligned with the third axis, $\mathbf{H}_+$ is a rotating vectors in a plane perpendicular to $\mathbf{H}_3$. The wave function $\Psi=(\psi_1,\psi_2)^{\mathrm{T}}$ is also used to define the state vector $\mathbf{s}$
\begin{align}
    \mathbf{s} =& \Psi^\dagger \frac{\boldsymbol{\sigma}}{2}\Psi \\
    =& \frac{1}{2}\begin{pmatrix}
    2\,\mathrm{Re}\,(\psi_1^* \psi_2) \\
    2\,\mathrm{Im}\,(\psi_1^*\psi_2) \\
    \lvert \psi_1 \rvert^2 - \lvert \psi_2 \rvert^2
    \end{pmatrix}
\end{align}
The third component of $\mathbf{s}$, which is denoted as $s_3$, is within range $[-1/2,1/2]$. The two limits, $s_3=-1/2$ and $s=1/2$ stand for the system in high energy state and low energy state respectively. $s_3=0$ if the system has equal probabilities to be on high energy state and low energy state. The Schr\"odinger equation becomes
\begin{equation}
\frac{\mathrm{d}}{\mathrm{d} t } \mathbf{s} = \mathbf{s} \times \mathbf{H},
\end{equation}
which is the precession equation. For static $\mathbf{H}$, the state vector $\mathbf{s}$ precess around $\mathbf{H}$. 

In a frame that corotates with $\mathbf{H}_+$, which is described in Fig.~\ref{fig-rabi-isospin-rotating-frame}, the new Hamiltonian is
\begin{equation}
\frac{\mathrm d}{\mathrm d t } \mathbf{s} = \mathbf{s} \times (\mathbf{H}'_3 + \mathbf{H}‘_+),
\end{equation}
where
\begin{equation}
\mathbf{H}'_3 = \begin{pmatrix}
    0 \\ 0 \\  \omega_{\mathrm{R}} - k_{\mathrm R}
    \end{pmatrix}, \qquad \mathbf{H}'_+ = \begin{pmatrix}
    A_{\mathrm{R}} \\ 0 \\  0
    \end{pmatrix}.
\end{equation}
The state vector $\mathbf{s}$ precess around a static vector $\mathbf{H}'_3 + \mathbf{H}'_+$ with a frequency $\Omega_{\mathrm R} = \sqrt{ \lvert A_{\mathrm{R}}\rvert^2 + (k_{\mathrm{R}} - \omega_{\mathrm R})^2 }$. A geometric analysis by projecting the state vector $\mathbf{s}$ on to the verticle axis shows that
\begin{equation}
s_3 = \frac{1}{2} - \frac{\lvert A_{\mathrm R}\rvert ^2}{\Omega_{\mathrm R}^2}\sin^2\left(\frac{\Omega_{\mathrm R}}{2} t\right).
\end{equation}
Resonance occurs when the term $\mathbf{H}_3$ disappears in this corotating frame, since the state vector $\mathbf{s}$ converts completely between $+1/2$ (low energy state) and $-1/2$ (high energy state).



Such a system has analytical transition probability from low energy state to high energy state
\begin{equation}
    P(r) = \frac{1}{2}(1- 2 s_3(r))= \frac{\left \lvert A_{\mathrm{R}} \right \rvert ^2}{ \Omega_{\mathrm R}^2 } \sin^2 \left( \frac{\Omega_{\mathrm R}}{2} t \right),
    \label{rabi-system-transition-probability}
\end{equation}
where
\begin{equation}
\Omega_{\mathrm R} = \sqrt{ \lvert A_{\mathrm{R}}\rvert^2 + (k_{\mathrm{R}} - \omega_{\mathrm R})^2 }
\end{equation} is known as Rabi frequency. The detuning, which is defined by $k_{\mathrm{R}} - \omega_{\mathrm R}$, determines how off-resonance the system is, and amplitude of driving field $A_{\mathrm{R}}$ determines the resonance width,
\begin{align}
\text{Detuning} =&~\lvert k_{\mathrm{R}} - \omega_{\mathrm R} \rvert, \\
\text{Resonance Width} =&~\lvert A_{\mathrm R} \rvert.
\end{align}
The transition probability oscillates with frequency $\Omega_{\mathrm R}$. However, the amplitude $A_1$ is the dominate factor for oscillation frequency when the system is close to resonance. The phase of the matter potential $\phi_{\mathrm{R}}$ has no effect on the transition probability since it only determines the initial phase of driving Hamiltonian vector $\mathbf{H}_+$. We also notice that the transition amplitude is determined by relative detuning $Q$, which is defined as
\begin{equation}
    Q = \left\lvert \frac{ k_{\mathrm R} - \omega_{\mathrm R}}{A_{\mathrm R}} \right\rvert.
\end{equation}


Given a Rabi oscillation system with two driving frequencies
\begin{align*}
    H_{\mathrm R} =& -\frac{\omega_{\mathrm R}}{2}\sigma_3 - \frac{A_{1} }{2}  \left( \cos(k_{1} t +\phi_{1})\sigma_1  - \sin(k_{1} t +\phi_{1}) \sigma_2\right) \nonumber\\
    & - \frac{A_{2} }{2}  \left( \cos(k_{2} t +\phi_{2})\sigma_1  - \sin(k_{2} t +\phi_{2}) \sigma_2\right)
\end{align*}
we decompose it into $\mathbf{H}_{\mathrm R}=\mathbf{H}_3 + \mathbf{H}_{1} + \mathbf{H}_2$, where
\begin{equation*}
    \mathbf{H}_1 =  \begin{pmatrix}
     A_{1} \cos(k_{1}t+\phi_{1}) \\
     A_{1} \sin(k_{1}t+\phi_{1})  \\
     0
      \end{pmatrix},   \mathbf{H}_2 =  \begin{pmatrix}
     A_{2} \cos(k_{2}t+\phi_{2}) \\
     A_{2} \sin(k_{2}t+\phi_{2})  \\
     0
      \end{pmatrix}.
\end{equation*}
Assume $\mathbf{H}_1$ is the on-resonance perturbation which requires $k_1 = \omega_{\mathrm{R}}$. The most general condition that we can drop the new perturbation $\mathbf{H}_2$ is to make sure $k_2$ is far from the resonance condition compared to the resonance width,
\begin{equation}
Q \equiv \frac{\lvert k_2 -\omega_{\mathrm R}\rvert}{\lvert A_2\rvert} \gg 1.
\end{equation}
A more applicable condition is derived in Sec.~\ref{sec:interference-with-long-wavelength-mode}.



\bibliographystyle{apsrev4-1}
\bibliography{ref.bib} 



%%%%%%%%%%%%%%%%%%%%%%%%%%%%%%%%%%%%%%%%%%
%%%%%%%%%%%%% APPENDIX  %%%%%%%%%%%%%%%%%%
%%%%%%%%%%%%%%%%%%%%%%%%%%%%%%%%%%%%%%%%%%









\end{document}