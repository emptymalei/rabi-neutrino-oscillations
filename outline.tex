% ****** Start of file apssamp.tex ******
%
%   This file is part of the APS files in the REVTeX 4.1 distribution.
%   Version 4.1r of REVTeX, August 2010
%
%   Copyright (c) 2009, 2010 The American Physical Society.
%
%   See the REVTeX 4 README file for restrictions and more information.
%
% TeX'ing this file requires that you have AMS-LaTeX 2.0 installed
% as well as the rest of the prerequisites for REVTeX 4.1
%
% See the REVTeX 4 README file
% It also requires running BibTeX. The commands are as follows:
%
%  1)  latex apssamp.tex
%  2)  bibtex apssamp
%  3)  latex apssamp.tex
%  4)  latex apssamp.tex
%
\documentclass[%
%reprint,
%superscriptaddress,
%groupedaddress,
%unsortedaddress,
%runinaddress,
%frontmatterverbose, 
preprint,
%showpacs,preprintnumbers,
%nofootinbib,
%nobibnotes,
%bibnotes,
 amsmath,amssymb,
 aps,
%pra,
%prb,
%rmp,
%prstab,
%prstper,
%floatfix,
]{revtex4-1}

\usepackage{graphicx}% Include figure files
\usepackage{dcolumn}% Align table columns on decimal point
\usepackage{bm}% bold math
%\usepackage{hyperref}% add hypertext capabilities
%\usepackage[mathlines]{lineno}% Enable numbering of text and display math
%\linenumbers\relax % Commence numbering lines

%\usepackage[showframe,%Uncomment any one of the following lines to test 
%%scale=0.7, marginratio={1:1, 2:3}, ignoreall,% default settings
%%text={7in,10in},centering,
%%margin=1.5in,
%%total={6.5in,8.75in}, top=1.2in, left=0.9in, includefoot,
%%height=10in,a5paper,hmargin={3cm,0.8in},
%]{geometry}


%%%%%%%%%%%%%%%%%%%%%%%%%%%
%%%%%%  PREAMBLES %%%%%%%%%
\newcommand{\ud}[1]{{#1^{\dagger}}}
\newcommand{\bra}[1]{\left\langle #1\right|}
\newcommand{\ket}[1]{\left| #1\right\rangle}
\newcommand\Tr{\mathrm{Tr}}
\newcommand{\braket}[2]{\langle #1 \mid #2 \rangle}
\newcommand\I{\mathbb{I}}
\newcommand{\avg}[1]{\left< #1 \right>}
%%%%%%  PREAMBLES %%%%%%%%%
%%%%%%%%%%%%%%%%%%%%%%%%%%%

\begin{document}

%\preprint{APS/123-QED}

\title{Stimulated Neutrino Oscillations - A Rabi Oscillations View}% Force line breaks with \\
%\thanks{A footnote to the article title}%

\author{}
%  \altaffiliation[Also at ]{Physics Department, XYZ University.}%Lines break automatically or can be forced with \\
% \author{Second Author}%
%  \email{Second.Author@institution.edu}
% \affiliation{%
%  Authors' institution and/or address\\
%  This line break forced with \textbackslash\textbackslash
% }%

% \collaboration{MUSO Collaboration}%\noaffiliation

% \author{Charlie Author}
%  \homepage{http://www.Second.institution.edu/~Charlie.Author}
% \affiliation{
%  Second institution and/or address\\
%  This line break forced% with \\
% }%
% \affiliation{
%  Third institution, the second for Charlie Author
% }%
% \author{Delta Author}
% \affiliation{%
%  Authors' institution and/or address\\
%  This line break forced with \textbackslash\textbackslash
% }%

% \collaboration{CLEO Collaboration}%\noaffiliation

\date{\today}% It is always \today, today,
             %  but any date may be explicitly specified

\begin{abstract}
ABSTRACT PLACEHOLDER
% \begin{description}
% \item[Usage]
% Secondary publications and information retrieval purposes.
% \item[PACS numbers]
% May be entered using the \verb+\pacs{#1}+ command.
% \item[Structure]
% You may use the \texttt{description} environment to structure your abstract;
% use the optional argument of the \verb+\item+ command to give the category of each item. 
% \end{description}
\end{abstract}

% \pacs{Valid PACS appear here}% PACS, the Physics and Astronomy
                             % Classification Scheme.
%\keywords{Suggested keywords}%Use showkeys class option if keyword
                              %display desired
\maketitle

%\tableofcontents

\section{\label{introduction}Introduction}

\begin{enumerate}
    \item Work done before, but the physics is not clear
    \item Decompose the system into Rabi oscillations
\end{enumerate}


\section{\label{review}Brief Review of this Topic}


Matter profile

\begin{equation}
    \lambda(x) = \lambda_0 + \sum_{n=1}^{N} \delta \lambda_n (x),
\end{equation}

where

\begin{align}
    \lambda_0 &= \sqrt{2}G_F n_{e0} \\
    \delta \lambda_n(x) &= \sqrt{2}G_F \delta n_{e,n}(x)
\end{align}


Hamiltonian becomes

\begin{equation}
    H^{(m)} = - \frac{\omega_m}{2} \sigma_3 + \frac{\delta \lambda}{2} \cos 2\theta_m \sigma_3 - \frac{\delta \lambda}{2} \sin 2 \theta_m \sigma_1.
\end{equation}

Apply rotation


\begin{equation}
\begin{pmatrix} \ket{\nu_1} \\ \ket{\nu_2} \end{pmatrix} = \begin{pmatrix} e^{-i \eta (x)} & 0 \\  0 & e^{i \eta (x)}  \end{pmatrix} \begin{pmatrix} \ket{\nu_{b1}} \\ \ket{\nu_{b2}} \end{pmatrix}.
\end{equation}

Hamiltonian

\begin{equation}
 H = -\frac{\sigma_3}{2} - \frac{\delta \lambda}{2} \sin 2\theta_m \begin{pmatrix} 0 & e^{2i\eta(x)} \\ e^{-2 i\eta(x) } & 0 \end{pmatrix}  \begin{pmatrix} \psi_{b1} \\ \psi_{b2} \end{pmatrix},
\end{equation}

with

\begin{equation}
 \eta(x) =  \frac{\cos 2\theta_m}{2} \int_0^x \delta\lambda (\tau) d\tau.
\end{equation}




\subsection{Modes Method}

\begin{align}
H =& -\frac{\omega_m}{2} \sigma_3 \nonumber \\
&+ \frac{1}{2} \sum_{n_1} \cdots \sum_{n_N} \begin{pmatrix} 0 & B_{n_1,\cdots,n_N} \Phi_{n_1,\cdots, n_N} e^{i \left( \sum_{a} n_a k_a   \right)x} \\ B_{n_1,\cdots,n_N}^* \Phi_{n_1,\cdots, n_N}^* e^{-i \left( \sum_{a} n_a k_a   \right)x} & 0 \end{pmatrix}.
\end{align}


Single perturbation modes (width at large n limit)

\begin{figure}
    \centering
    \includegraphics[width=\textwidth]{assets/placeholder.jpg}
    \caption{Single perturbation: Resonance, modes, and width of each mode}
    \label{fig:single-perturbation}
\end{figure}





\subsection{Rabi Oscillation}



\begin{equation}
H_{\mathrm R} = -\frac{\omega_m}{2} \sigma_3 - A \cos(k t) \sigma_1
\end{equation}

Important mode

\begin{align}
H_{\mathrm R}' &= -\frac{\omega_m}{2} \sigma_3 - \frac{A}{2} \begin{pmatrix}0 & e^{i k t} \\ e^{-i k t} & 0 \end{pmatrix} \\
& =  -\frac{\omega_m}{2} \sigma_3 - \frac{A}{2} \cos(kt) \sigma_1 + \frac{A}{2} \sin (kt) \sigma_2.
\end{align}

Probability

\begin{equation}
P(x) = \frac{\lvert A\rvert^2}{ \lvert A\rvert^2 + (k - \omega_m)^2 }  \sin^2 \left( \sqrt{ \lvert A\rvert^2 + (k - \omega_m)^2 } x/2 \right).
\end{equation}





\section{Destruction Effect}

Construct system of two perturbations

\begin{equation}
-\frac{\omega_m}{2} \sigma_3 - \frac{1}{2} \left(A_1 \cos(k_1 t) + A_2 \cos(k_2t) \right) \sigma_1 + \frac{1}{2} \left( A_1 \sin (k_1t) + A_2 \cos(k_2t) \right) \sigma_2.
\end{equation}

with $k_1 \gg k_2$

\begin{equation}
\omega_m' = \sqrt{\omega_m^2 + A_2^2}
\end{equation}

Predict the critical $A_2$ that significantly reduces the transition amplitude,

\begin{equation}
\lvert k_1 - \omega_m' \rvert \gtrsim \text{width of resonance}.
\end{equation}

Figures showing that this hypothesis is true.

\begin{figure}[!htbp]
    \centering
    \includegraphics[width=\textwidth]{assets/placeholder.jpg}
    \caption{Reduction of transition amplitude. The mode at resonance; Adding a second mode could destroy the resonance if the condition is satisfied. }
    \label{fig:my_label}
\end{figure}



\clearpage
\appendix
\section{Keypoints}



\begin{itemize}
    \item A New Basis: Hamiltonian looks like a Rabi oscillation but with some complicated perturbations.
    \item Jacobi-Anger Expansion: Write the system into a superposition of Rabi oscillations.
    \item Each mode can be solved and explained using Rabi oscillation. Slightly different from Rabi oscillation but approximately true.
    \item Whether a mode is important depends on several factors.
        \begin{itemize}
            \item Width of resonance $B$
            \item Deviation from exact resonance $g$
            \item Oscillation wavelength of mode (related to $B$ and $g$ ) compared to size of physical system
        \end{itemize}
    \item Interference between each modes can cause destruction.
        \begin{itemize}
            \item Slow rotating perturbation
        \end{itemize}
    
\end{itemize}
    








\end{document}