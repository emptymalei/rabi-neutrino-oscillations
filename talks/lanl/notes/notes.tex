%!TEX program = pdflatex
\documentclass{tufte-handout}
%\geometry{showframe}% for debugging purposes -- displays the margins

\usepackage{amsmath}

% Set up the images/graphics package
\usepackage{graphicx}
\setkeys{Gin}{width=\linewidth,totalheight=\textheight,keepaspectratio}
\graphicspath{{assets/}}

%%%%%%%%%%%%%%%%%%%%%%%%%
%%%%% For Timeline %%%%%%

% http://tex.stackexchange.com/questions/196794/how-can-you-create-a-vertical-timeline

\usepackage[T1]{fontenc}
\usepackage[utf8]{inputenc}
\usepackage{charter}
\usepackage{environ}
\usepackage{tikz}
\usetikzlibrary{calc,matrix}

\makeatletter
\let\matamp=&
\catcode`\&=13
\makeatletter
\def&{\iftikz@is@matrix
  \pgfmatrixnextcell
  \else
  \matamp
  \fi}
\makeatother

\newcounter{lines}
\def\endlr{\stepcounter{lines}\\}

\newcounter{vtml}
\setcounter{vtml}{0}

\newif\ifvtimelinetitle
\newif\ifvtimebottomline
\tikzset{description/.style={
  column 2/.append style={#1}
 },
 timeline color/.store in=\vtmlcolor,
 timeline color=red!80!black,
 timeline color st/.style={fill=\vtmlcolor,draw=\vtmlcolor},
 use timeline header/.is if=vtimelinetitle,
 use timeline header=false,
 add bottom line/.is if=vtimebottomline,
 add bottom line=false,
 timeline title/.store in=\vtimelinetitle,
 timeline title={},
 line offset/.store in=\lineoffset,
 line offset=4pt,
}

\NewEnviron{vtimeline}[1][]{%
\setcounter{lines}{1}%
\stepcounter{vtml}%
\begin{tikzpicture}[column 1/.style={anchor=east},
 column 2/.style={anchor=west},
 text depth=0pt,text height=1ex,
 row sep=1ex,
 column sep=1em,
 #1
]
\matrix(vtimeline\thevtml)[matrix of nodes]{\BODY};
\pgfmathtruncatemacro\endmtx{\thelines-1}
\path[timeline color st] 
($(vtimeline\thevtml-1-1.north east)!0.5!(vtimeline\thevtml-1-2.north west)$)--
($(vtimeline\thevtml-\endmtx-1.south east)!0.5!(vtimeline\thevtml-\endmtx-2.south west)$);
\foreach \x in {1,...,\endmtx}{
 \node[circle,timeline color st, inner sep=0.15pt, draw=white, thick] 
 (vtimeline\thevtml-c-\x) at 
 ($(vtimeline\thevtml-\x-1.east)!0.5!(vtimeline\thevtml-\x-2.west)$){};
 \draw[timeline color st](vtimeline\thevtml-c-\x.west)--++(-3pt,0);
 }
 \ifvtimelinetitle%
  \draw[timeline color st]([yshift=\lineoffset]vtimeline\thevtml.north west)--
  ([yshift=\lineoffset]vtimeline\thevtml.north east);
  \node[anchor=west,yshift=16pt,font=\large]
   at (vtimeline\thevtml-1-1.north west) 
   {\textsc{Timeline \thevtml}: \textit{\vtimelinetitle}};
 \else%
  \relax%
 \fi%
 \ifvtimebottomline%
   \draw[timeline color st]([yshift=-\lineoffset]vtimeline\thevtml.south west)--
  ([yshift=-\lineoffset]vtimeline\thevtml.south east);
 \else%
   \relax%
 \fi%
\end{tikzpicture}
}


%%%% Timeline END %%%%%%%
%%%%%%%%%%%%%%%%%%%%%%%%%


\title{Neutrino Oscillations in Matter \thanks{Candidacy Exam Notes}}
\author[Lei Ma]{Lei Ma}
\date{\today}  % if the \date{} command is left out, the current date will be used


%%%%%%%%%%%%%%%%%%%%%%%%%%%%%%%%%%%%%%%%%%%%%%%
%%%%%%%% Tufte Style Requirement %%%%%%%%%%%%%%

% The following package makes prettier tables.  We're all about the bling!
\usepackage{booktabs}

% The units package provides nice, non-stacked fractions and better spacing
% for units.
\usepackage{units}

\usepackage{ulem}

% The fancyvrb package lets us customize the formatting of verbatim
% environments.  We use a slightly smaller font.
\usepackage{fancyvrb}
\fvset{fontsize=\normalsize}

% Small sections of multiple columns
\usepackage{multicol}

% code highlighting
\usepackage{listings}

% For url links
\usepackage{hyperref}


% \usepackage{minted}
% \usepackage[utf8]{inputenc}
% \usepackage[english]{babel}

% Provides paragraphs of dummy text

%%%%%%%%%% Tufte Style Requirement END %%%%%%%%
%%%%%%%%%%%%%%%%%%%%%%%%%%%%%%%%%%%%%%%%%%%%%%%




\begin{document}

\maketitle % this prints the handout title, author, and date

\begin{abstract}
\noindent Notes for my candidacy exam
\end{abstract}

\tableofcontents

\section{Outline}

The plan of the presentation is to give a short introduction about neutrino oscillation then neutrino oscillations with matter presence. The most important points are 


\begin{enumerate}
\item What do I want to find out
\item What have I done
\end{enumerate}


In the introduction, I should notice the audience about the importance of neutrino oscillations, here I address the neutrino oscillations in matter.

The story line should be

\begin{enumerate}
    \item What is neutrino?
    \item What is neutrino oscillation?
    \item Why do they oscillate?
    \item Why is neutrino oscillations important
    \item Math Related to neutrino oscillation
    \item Add in matter will it be easy
    \item Approach to oscillations in matter 
\end{enumerate}


%%%%%%%%% 
\section{Concepts  Questions}

\begin{itemize}
\item What are flavors?
\item What is exactly the interaction between neutrinos and matter?
\end{itemize}




\section{Introduction}


\subsection{What is Neutrino?}



\begin{vtimeline}[description={text width=10cm}, 
 row sep=4ex, 
 use timeline header,
 timeline title={(Partial) History of Neutrino}]
1930 & Pauli, letter to "Radioactive Ladies and Gentlemen"\endlr
1933 & Fermi, the name "neutrino" \endlr
1956 & Reines \& Cowan, first neutrino evidence \endlr
1957 & \textbf{Pontecorvo, theory of neutrino oscillations} \endlr
1968 & Homestake, first solar neutrino detection \endlr
1969 & Gribov \& Pontecorvo, solar neutrino oscillations \endlr
1978 \& 1985 & \textbf{Wolfenstein \& Mikheyev \& Smirnov, MSW effect} \endlr
1987 & Kamioka mine \& Morton salt mine, SN1987A neutrino\endlr
1998 \& 2001 & Super-Kamiokande \& SNO, solar neutrino oscillations \endlr
\end{vtimeline}


\begin{enumerate}
\item Pauli's letter postulated a neutral particle to explain the continuous beta spectrum. \url{http://microboone-docdb.fnal.gov/cgi-bin/RetrieveFile?docid=953;filename=pauli%20letter1930.pdf}
    
\end{enumerate}


\section{Tell the Story of Stimulated Neutrino Oscillations}


Questions expected

\begin{itemize}
    \item Is is really stimulated? Compared to what?
    \item Phase in matter profile is not related to the final result? I am not quite sure here because a redefinition of coordinate will cause two constant phase shift that have different signs in the two off-diagonal terms. 
    
    RWA in fact shows that the phase term is only the coupling between different orders of the effects.
    
    The numerical results are not really related to this phase. Probably an approximation. {\color{red}NEED SOME PROOF.}
    \item How to know hierarchy from solar neutrino experiment? Why not atmospheric neutrino experiments?
    
\end{itemize}



\subsection{How to Introduce the Solution I Had}

First of all, the Hamiltonian we are dealing with is

\begin{equation*}
\mathbf H = - \frac{\omega_m}{2}\boldsymbol{\sigma_3} + \frac{\delta \lambda}{2} \cos 2\theta_m \boldsymbol{\sigma_3} - \frac{\delta \lambda}{2} \sin \theta_m \boldsymbol{\sigma_1},
\end{equation*}

where the matter profile we used is

\begin{equation*}
    \lambda(x) = \lambda_0 + \delta \lambda(x).
\end{equation*}


I need to explain what basis I will use, a basis where the wave function is related to the background matter basis through

\begin{equation*}
    \begin{pmatrix} \psi_1 \\ \psi_2 \end{pmatrix} = \begin{pmatrix} e^{-i \eta (x)} & 0 \\  0 & e^{-i \eta (x)}  \end{pmatrix} \begin{pmatrix} \psi_{b1} \\ \psi_{b2} \end{pmatrix}.
\end{equation*}

Why? I need to remove the position dependence of the diagonal term. \footnote{
{\color{red} I am not actually sure what is the advantage of this basis. I just found it and it worked.}
}

I can solve out the unitary "rotation" matrix of the wave function.

\begin{equation*}
    \eta(x) - \eta(0) = - \frac{\omega_m}{2} x + \frac{\cos 2\theta_m}{2} \int_0^x \delta\lambda (\tau) d\tau.
\end{equation*}



\subsection{Rabi Oscillation}

Rabi oscillation Hamiltonian

\begin{equation*}
    \mathbf H = \begin{pmatrix}
    E_1 &  h \\
    h^* & E_2 \\
    \end{pmatrix}.
\end{equation*}







\bibliography{notes-ref}
\bibliographystyle{plainnat}


\end{document}