% ****** Start of file aipsamp.tex ******
%
%   This file is part of the AIP files in the AIP distribution for REVTeX 4.
%   Version 4.1 of REVTeX, October 2009
%
%   Copyright (c) 2009 American Institute of Physics.
%
%   See the AIP README file for restrictions and more information.
%
% TeX'ing this file requires that you have AMS-LaTeX 2.0 installed
% as well as the rest of the prerequisites for REVTeX 4.1
%
% It also requires running BibTeX. The commands are as follows:
%
%  1)  latex  aipsamp
%  2)  bibtex aipsamp
%  3)  latex  aipsamp
%  4)  latex  aipsamp
%
% Use this file as a source of example code for your aip document.
% Use the file aiptemplate.tex as a template for your document.
\documentclass[%
 aip,
%jmp,%
%bmf,%
%sd,%
rsi,%
 amsmath,amssymb,
preprint,%
%  reprint,%
%author-year,%
%author-numerical,%
]{revtex4-1}

\usepackage{graphicx}% Include figure files
\usepackage{dcolumn}% Align table columns on decimal point
\usepackage{bm}% bold math
%\usepackage[mathlines]{lineno}% Enable numbering of text and display math
%\linenumbers\relax % Commence numbering lines

\begin{document}

\preprint{}

\title[Some Notes]{Some Notes}% Force line breaks with \\
\thanks{}

\author{Lei Ma}
 
\date{\today}% It is always \today, today,
             %  but any date may be explicitly specified

\begin{abstract}
Just some note

\end{abstract}

\maketitle



\section{Some Notes}


\subsection{Mathematical Proof that Jacobi-Anger Expansion Brings Us the Expansion onto Many Constant Matter Perturbations}


We can prove it by using a inverse transformation. However, we can make this clear by the special example that $\delta(x)=A$ which is a constant matter potential on top of the background potential.

The Equation of motion becomes

\begin{equation}
\mathrm i \frac{d}{dx}\begin{pmatrix}
\psi_{r1}\\
\psi_{r2}
\end{pmatrix} =
\left[ -\frac{\omega_{\mathrm m}}{2} \sigma_3 - \frac{A}{2} \sin 2\theta_{\mathrm m} \begin{pmatrix}
0 & e^{\mathrm i \cos 2\theta_{\mathrm m} A x } \\
 e^{-\mathrm i \cos 2\theta_{\mathrm m} A x } & 0
\end{pmatrix} \right] \begin{pmatrix}
\psi_{r1}\\
\psi_{r2}
\end{pmatrix}.
\end{equation}


By comparing this equation with the Jacobi-Anger expanded equation, we find that to interpret each Jacobi-Anger expanded mode as constant matter potential perturbation requires

\begin{align}
B_n = & A\sin 2\theta_{\mathrm m} \\
nk_1 = & A\cos 2\theta_{\mathrm m}.
\end{align}

To find out the relation, we need to solve $A$ and $\theta_{\mathrm m}$,

\begin{align}
\theta_{\mathrm m} =& \arctan \frac{B_n}{n k_1} \\
A =& n k_1 /\cos 2\theta_{\mathrm m}.
\end{align}

So each mode corresponds to a constant matter potential perturbation with amplitude $nk_1/\cos 2\theta_{\mathrm m}$ and also a particular mixing angle $\theta_{\mathrm m}$.

However, for the purpose of the paper, I believe we could simply interpret the purpose of Jacobi-Anger expansion as expansion on to systems with infinite constant matter perturbations.








\end{document}
%
% ****** End of file aipsamp.tex ******